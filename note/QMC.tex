\documentclass[UTF8,cs4size]{ctexart}
\usepackage[top =2.54cm ,bottom =2.54cm ,left =3.18cm ,right =3.18 cm]{ geometry}
\usepackage{amsmath}
\usepackage{color}
\usepackage{graphicx}
\graphicspath{{graphics/}}
\bibliographystyle{plain}
\def\ee{\mathrm e}
\def\dd{\mathrm d}
\def\rr{\mathbf r}
\def\xx{\mathbf x}
\def\kk{\mathbf k}
\title{Hubbard Model(Fermion) 的 QMC}
\author{}
\date{}
\begin{document}
\maketitle
\zihao{-4}
首先我们考虑 Hubbard Model 的巨正则哈密顿量:
\[
    H=-t \sum_{<i,j>,\sigma} c^{\dag}_{i,\sigma}c_{j,\sigma} + U \sum_{i} n_{i\uparrow} n_{i\downarrow}
    -\mu \sum_{i,\sigma} n_{i,\sigma}
\]
对应的 Hubbard Model 的巨配分函数为:
\[
    Z=\mathcal{T}r \  \ee^{-\beta H}   
\]
考虑到哈密顿量各项之间并不对易,我们要利用Suzuki-Trotter分解。Suzuki-Trotter分解利用到的事实就是:
\[
    \ee^{\Delta \tau (A+B)}=\ee^{\Delta \tau A} \ee^{\Delta \tau B} +O(\Delta \tau^2 [A,B])    
\]
Suzuki-Trotter分解即为:
\[
    \ee^{-\beta (K+V)}  = (\ee^{-\Delta \tau (K+V)})^M =(\ee^{-\Delta \tau K} \ee^{-\Delta \tau V})^M
    +O(\Delta \tau^2)
\]
这与路径积分相似,将虚时$(0,\beta)$分成$M$段。
对于哈密顿量中的相互作用项,我们将使用 Hubbard-Stratonovich 变换。其变换形式如下:
\[
    \ee^{-\Delta \tau U n_{\uparrow} n_{\downarrow}} =\frac{1}{2} \ee^{-\frac{\Delta \tau U}{2}n}
    \sum_{s=\pm 1} \ee^{-sm\lambda} = \frac{1}{2} \sum_{s=\pm 1} \prod_{\sigma = \uparrow \downarrow}
    \ee^{-\left(\sigma s \lambda + \frac{\Delta \tau U}{2}\right)n_{\sigma}}    
\]
其中$n=n_{\uparrow} + n_{\downarrow}$,$m=n_{\uparrow} - n_{\downarrow}$,$\lambda$的值满足下列方程
\[
    \cosh \lambda = \ee^{\Delta \tau U /2}    
\]
这个HS变换很容易验证,因为对于费米子系统上述算符的希尔伯特空间是四维的,我们只需要验证左边的算符与右边的算符
作用在希尔伯特空间的基矢上得到相同的结果。\\
\textcircled{1}$\left| n_{\uparrow}=0,n_{\downarrow}=0 \right>$
\[
    \begin{aligned}
        lhs&=1\\
        rhs&=\frac{1}{2}(1+1)=1
    \end{aligned}    
\]
\textcircled{2}$\left| n_{\uparrow}=0,n_{\downarrow}=1 \right>$
\[
    \begin{aligned}
        lhs&=1\\
        rhs&=\frac{1}{2}\ee^{-\frac{\Delta \tau U}{2}}\left(\ee^{\lambda} + \ee^{-\lambda}\right)
        =\ee^{-\frac{\Delta \tau U}{2}} \cosh \lambda =1
    \end{aligned}    
\]
\textcircled{3}$\left| n_{\uparrow}=1,n_{\downarrow}=0 \right>$
\[
    \begin{aligned}
        lhs&=1\\
        rhs&=\frac{1}{2}\ee^{-\frac{\Delta \tau U}{2}}\left(\ee^{-\lambda} + \ee^{\lambda}\right)
        =\ee^{-\frac{\Delta \tau U}{2}} \cosh \lambda =1
    \end{aligned}    
\]
\textcircled{4}$\left| n_{\uparrow}=1,n_{\downarrow}=1 \right>$
\[
    \begin{aligned}
        lhs&=\ee^{-\Delta \tau U}\\
        rhs&=\frac{1}{2}\ee^{-\Delta \tau U}\left(1+1\right) = \ee^{-\Delta \tau U}
    \end{aligned}    
\]
经过Hubbard-Stratonovich变换之后我们可以把配分函数写成如下形式:
\[
    Z=\left( \frac{1}{2}\right)^{L^d M} \mathop{\mathrm{Tr}} \limits_{\{s\}} \mathcal{T}r
    \prod_{\sigma=\uparrow , \downarrow} \prod_{l=M}^{1} 
    \ee^{-\Delta \tau \sum_{i,j} c^{\dag}_{i,\sigma} K_{ij} c_{j,\sigma}}
    \ee^{-\Delta \tau \sum_{i,j} c^{\dag}_{i,\sigma} V^{\sigma}_{ij}(l) c_{j,\sigma}}
\]
其中
\[
    K_{ij}=
    \left\{
        \begin{aligned}
            &-t &\text{if i and j are nearest neighbours}\\
            &0  &\text{otherwise}
        \end{aligned}
    \right.   
\]
\[
    V^{\sigma}_{ij}(l)=\left[\frac{1}{\Delta \tau} \sigma s_{i}(l) \lambda 
        +\left(\frac{U}{2} - \mu\right) \right] \delta_{ij}
\]
利用\ref{bookproof}的命题三,我们将费米子算符的trace求出。则配分函数为:
\[
    Z=\left( \frac{1}{2}\right)^{L^d M} \mathop{\mathrm{Tr}} \limits_{\{s\}}
    \prod_{\sigma=\uparrow , \downarrow}
    \det\left[1+B^{\sigma}_{M} \cdots B^{\sigma}_{2} B^{\sigma}_{1} \right]
\]
其中:
\[
    B^{\sigma}_{l}=\ee^{-\Delta \tau K} \ee^{-\Delta \tau V^{\sigma}(l)}
\]
由于$V$矩阵是对角的,我们可以直接计算出$B$矩阵以及$B^{-1}$矩阵:
\[
    \begin{aligned}
        \left[B ^ {\sigma}_{l}\right]_{ij} &= [\ee^{-\Delta \tau K}]_{ij}[\ee^{-\Delta \tau V^{\sigma}(l)}]_{jj}\\
        &= [\ee^{-\Delta K}]_{ij} \ee^{-\Delta \tau
            \left[\frac{\sigma s_{j}(l)\lambda}{\Delta \tau} + \left(\frac{U}{2} -\mu \right)\right]}
    \end{aligned}
\]
\[
    \begin{aligned}
        \left\{\left[B ^ {\sigma}_{l}\right]^{-1}\right\}_{ij} &=
            [\ee^{\Delta \tau V^{\sigma} (l)}]_{ii} [\ee^{\Delta K}]_{ij} \\
        &=  \ee^{\Delta \tau
        \left[\frac{\sigma s_{i}(l)\lambda}{\Delta \tau} + \left(\frac{U}{2} -\mu \right)\right]}
        [\ee^{\Delta K}]_{ij}
    \end{aligned}  
\]
定义
\[
    O^{\sigma}(\{s\})= 1+ B^{\sigma}_{M} \cdots B^{\sigma}_{2} B^{\sigma}_{1}
\]
则
\[
    \begin{aligned}
    Z&=\left( \frac{1}{2}\right)^{L^d M} \mathop{\mathrm{Tr}} \limits_{\{s\}}
    \det O^{\uparrow}(\{s\}) \cdot \det O^{\downarrow} (\{s\})\\
    &= \mathop{\mathrm{Tr}} \limits_{\{s\}} \rho (\{s\})
    \end{aligned}
\]
最后一个等号可以认为是辅助场的等效密度矩阵的定义。可以证明在half-filling$(\mu=\frac{U}{2}$)时,$\rho (\{s\})>0$。
首先,我们引入电子空穴变换,即
\[
    \begin{aligned}
        d_{i,\sigma} &= (-1)^{i} c^{\dag}_{i,\sigma}\\
        d^{\dag}_{i,\sigma} &= (-1)^{i} c_{i,\sigma}
    \end{aligned}   
\]
则 
\[
    \tilde{n}_{i,\sigma} =  d^{\dag}_{i,\sigma} d_{i,\sigma} =   c_{i,\sigma} c^{\dag}_{i,\sigma}
    =1- n_{i,\sigma}
\]
从而
\[
    \begin{aligned}
        \det O^{\uparrow}(\{s\}) &= \mathcal{T}r \prod_{l=M}^{1} 
        \ee^{-\Delta \tau \sum_{i,j} c^{\dag}_{i} K_{ij} c_{j}}
        \ee^{- \lambda \sum_{i} c^{\dag}_{i} s_{i}(l) c_{i}}\\
        &=\mathcal{T}r \prod_{l=M}^{1} 
        \ee^{-\Delta \tau \sum_{i,j} d^{\dag}_{i} K_{ij} d_{j}}
        \ee^{- \lambda \sum_{i} s_{i}(l) (1-d^{\dag}_{i}d_{i})}\\
        &=\mathcal{T}r \prod_{l=M}^{1} 
        \ee^{-\Delta \tau \sum_{i,j} d^{\dag}_{i} K_{ij} d_{j}}
        \ee^{ \lambda \sum_{i} s_{i}(l) d^{\dag}_{i}d_{i}} \ee^{-\lambda \sum_{i} s_{i}(l)}\\
        &= \ee^{-\lambda \sum_{i,l} s_{i}(l)} \det O^{\downarrow} (\{s\})
    \end{aligned}    
\]
故$\det O^{\uparrow}(\{s\}) \cdot \det O^{\downarrow} (\{s\}) > 0$。

现在我们来考虑一些物理量的期待值:\\
考虑算符$O_\sigma=\mathbf{c^{\dag}_{\sigma}} A \mathbf{c_{\sigma}}$
\[
    \left< O_\sigma(l) \right> = \frac{ \mathrm{Tr}[ \left< O_\sigma (l) \right>_{\{s\}} \rho(\{s\})]}
    {\mathrm{Tr}[ \rho(\{s\})]} 
\]
\[
    \begin{aligned}
        \left< O_\sigma (l) \right>_{\{s\}} &=
        \frac{\partial}{\partial \eta} \ln \mathrm{Tr} \left.\left[D_{M}^{\sigma} \cdots D_{l+1}^{\sigma} 
        \ee^{\eta O_\sigma} D_{l}^{\sigma} \cdots D_{1}^{\sigma} \right] 
        \right|_{\eta=0}\\
        &=  \frac{\partial}{\partial \eta} \ln \det \left. \left[1+B_{M}^{\sigma} \cdots B_{l+1}^{\sigma} 
        \ee^{\eta O_\sigma} B_{l}^{\sigma} \cdots B_{1}^{\sigma} \right] \right|_{\eta=0}\\
        &=  \frac{\partial}{\partial \eta} \mathrm{Tr} \ln \left. \left[1+B_{M}^{\sigma} \cdots B_{l+1}^{\sigma} 
        \ee^{\eta O_\sigma} B_{l}^{\sigma} \cdots B_{1}^{\sigma} \right] \right|_{\eta=0}\\
        &=\mathrm{Tr} \left[B_{l}^{\sigma} \cdots B_{1}^{\sigma} \left(
        1+B_{M}^{\sigma} \cdots  B_{1}^{\sigma}\right)^{-1} B_{M}^{\sigma} \cdots B_{l+1}^{\sigma} O_\sigma
        \right]\\
        &=\mathrm{Tr} \left[\left(1- \left(1+B_{l}^{\sigma} \cdots B_{1}^{\sigma}
        B_{M}^{\sigma} \cdots  B_{l+1}^{\sigma} \right)^{-1} \right) O_\sigma\right]
    \end{aligned}    
\]
上式用到了$B\left( 1 + A B \right)^{-1} A = 1 - \left( 1+ B A \right)^{-1}$如果$A,B$是是可交换的,
那么这是显然的,事实上,对于$A,B$不对易的情况这也是对的:
\[
    \begin{aligned}
          &B\left( 1 + A B \right)^{-1} A  \\ 
        = &\left[ A^{-1} \left( 1 + A B \right) B^{-1} \right] ^{-1} \\
        = &\left( A^{-1} B^{-1} + 1 \right)^{-1} \\
        = &\left( A^{-1} B^{-1} + 1 \right)^{-1}
            \left(A^{-1} B^{-1} + 1  - A^{-1} B^{-1} \right) \\
        = &1 - \left[ BA\left( A^{-1} B^{-1} + 1 \right) \right]^{-1} \\
        = &1 - \left( 1 + BA\right)^{-1}
    \end{aligned}
\]
上上式中$D_{l}^{\sigma}=\ee^{-\Delta \tau \sum_{i,j} c^{\dag}_{i,\sigma} K_{ij} c_{j,\sigma}}
\ee^{-\Delta \tau \sum_{i,j} c^{\dag}_{i,\sigma} V^{\sigma}_{ij}(l) c_{j,\sigma}}$。
很容易可以验证
\[
    \left< c_{i,\sigma}(l) c^{\dag}_{j,\sigma}(l) \right>_{\{s\}} 
    = \left[\left(1+B_{l}^{\sigma} \cdots B_{1}^{\sigma}
    B_{M}^{\sigma} \cdots  B_{l+1}^{\sigma} \right)^{-1} \right]_{ij}
\]
接下来我们考虑$\left< c_{i,\sigma}(l_1) c^{\dag}_{j,\sigma}(l_2) \right>_{\{s\}}$,其中$l_1>l_2$
\[
    \begin{aligned}
    \left< c_{i,\sigma}(l_1) c^{\dag}_{j,\sigma}(l_2) \right>_{\{s\}}
    &=\frac{\mathrm{Tr} [D^{\sigma}_{M} \cdots D^{\sigma}_{l_1 +1} c_{i,\sigma}
    D^{\sigma}_{l_1} \cdots D^{\sigma}_{l_2 +1} c^{\dag}_{j,\sigma} 
    D^{\sigma}_{l_2} \cdots D^{\sigma}_{1}  ]}
    {\mathrm{Tr} [D^{\sigma}_{M} \cdots D^{\sigma}_{1}]}\\
    &=\frac{\mathrm{Tr} [D^{\sigma}_{M} \cdots D^{\sigma}_{l_2 +1}
    (D^{\sigma}_{l_1} \cdots D^{\sigma}_{l_2 +1})^{-1} c_{i,\sigma}
    D^{\sigma}_{l_1} \cdots D^{\sigma}_{l_2 +1} c^{\dag}_{j,\sigma} 
    D^{\sigma}_{l_2} \cdots D^{\sigma}_{1}  ]}
    {\mathrm{Tr} [D^{\sigma}_{M} \cdots D^{\sigma}_{1}]}
    \end{aligned}
\]
考察
\[
    c_{i}(\tau) =\ee^{\tau \mathbf{c^{\dag}} A \mathbf{c}}  c_{i}  
    \ee^{-\tau \mathbf{c^{\dag}} A \mathbf{c}}
\]
\[
    \frac{\partial c_{i}}{\partial \tau}   =  \ee^{\tau \mathbf{c^{\dag}} A \mathbf{c}}
    [\mathbf{c^{\dag}} A \mathbf{c},c_{i} ] \ee^{-\tau \mathbf{c^{\dag}} A \mathbf{c}}
    = -\sum_{j} A_{ij} c_{j}(\tau)
\]
故
\[
    c_{i}(\tau) = \sum_{j} \left[\ee^{-A \tau}\right]_{ij} c_{j}
\]
故
\[
    (D^{\sigma}_{l_1} \cdots D^{\sigma}_{l_2 +1})^{-1} c_{i,\sigma}
    D^{\sigma}_{l_1} \cdots D^{\sigma}_{l_2 +1}
    = \sum_{k} [B^{\sigma}_{l_1} \cdots B^{\sigma}_{l_2 +1}]_{ik} c_{k,\sigma}
\]
从而
\[
    \begin{aligned}
    \left< c_{i,\sigma}(l_1) c^{\dag}_{j,\sigma}(l_2) \right>_{\{s\}}
    &=\sum_{k} [B^{\sigma}_{l_1} \cdots B^{\sigma}_{l_2 +1}]_{ik} 
    \left< c_{k,\sigma}(l_1) c^{\dag}_{j,\sigma}(l_2) \right>_{\{s\}}\\
    &=\left[B^{\sigma}_{l_1} \cdots B^{\sigma}_{l_2 +1}
    \left(1+B_{l}^{\sigma} \cdots B_{1}^{\sigma}
    B_{M}^{\sigma} \cdots  B_{l+1}^{\sigma} \right)^{-1}
    \right]_{ij}
    \end{aligned}
\]
现在考察翻转自旋的概率。假如我们试图翻转$l$时间片段$i$格点上的Ising自旋即$s_{i}(l) \rightarrow -s_{i}(l)$
,这引起的变化为
\[
    V^{\sigma}_{ii}(l) =\frac{1}{\Delta \tau} \sigma \lambda s_{i}(l)
    \rightarrow
    \tilde{V}^{\sigma}_{ii}(l) = -\frac{1}{\Delta \tau} \sigma \lambda s_{i}(l)
\]
这将会导致$B^{\sigma}_{l} \rightarrow \tilde{B}^{\sigma}_{l} = B^{\sigma}_{l} \Delta^{\sigma}_{l}(i)$
,其中$\Delta^{\sigma}_{l}(i)$的各个矩阵元为
\[
    [\Delta^{\sigma}_{l}(i)]_{jk}=
    \left\{
        \begin{aligned}
            &0 &\text{if j$\neq$k} \\
            &1 &\text{if j=k$\neq$i}\\
            &\ee^{2\sigma \lambda s_{i}(l)} &\text{if j=k=i}
        \end{aligned}
    \right. 
\]
新旧构型的概率比为
\[
    r=\frac{\det O^{\uparrow}(\{s\}') \cdot \det O^{\downarrow} (\{s\}')}
    {\det O^{\uparrow}(\{s\}) \cdot \det O^{\downarrow} (\{s\})}
    =R_{\uparrow} R_{\downarrow}
\]
其中:
\[
    \begin{aligned}
        R_{\sigma} &=\frac{\det O^{\sigma}(\{s\}')}{\det O^{\sigma}(\{s\})}\\
        &=\frac{\det [1+B^{\sigma}_{M} \cdots B^{\sigma}_{l+1} B^{\sigma}_{l} 
        \Delta^{\sigma}_{l}(i) B^{\sigma}_{l-1} \cdots B^{\sigma}_{1}]}
        {\det [1+B^{\sigma}_{M} \cdots B^{\sigma}_{1}]}\\
        &=\frac{\det [1+A^{\sigma} (l) \Delta^{\sigma}_{l}(i)]}
        {\det [1+ A^{\sigma}(l)]}\\
        &=\det \left[\left[1+ \left([g^{\sigma}(l)]^{-1} -1 \right) \Delta^{\sigma}_{l}(i)
        \right] g^{\sigma} (l)\right]\\
        &=\det [[g^{\sigma} (l)]^{-1} \left[g^{\sigma} (l) +\left( 1 - g^{\sigma}(l) \right)\Delta^{\sigma}_{l}(i) \right] g^{\sigma} (l)]\\
        &=\det[1+(1-g^{\sigma}(l))(\Delta^{\sigma}_{l}(i)-1)]\\
        &=1+(1-[g^{\sigma}(l)]_{ii})(\ee^{2\sigma \lambda s_{i}(l)}-1)
    \end{aligned}
\]
其中$A^{\sigma}(l)$和$g^{\sigma}(l)$的定义为:
\[
    \begin{aligned}
        A^{\sigma} (l) &= B^{\sigma}_{l-1} \cdots B^{\sigma}_{1} B^{\sigma}_{M} \cdots B^{\sigma}_{l}\\
        g^{\sigma} (l) &= [1+A^{\sigma}(l)]^{-1}
    \end{aligned}    
\]
继续考虑$g$矩阵的更新:
\[
    \begin{aligned}
        \left[ \tilde{g}^{\sigma} (l) \right]^{-1} &= 1 + \tilde{A}^{\sigma}(l) \\
        &= 1 + A^{\sigma}(l) \Delta^{\sigma}_{l}(i) + A^{\sigma}(l) - A^{\sigma}(l) \\
        &= 1 + A^{\sigma}(l) + A^{\sigma}(l) \left(\Delta^{\sigma}_{l} (i) - 1 \right) \\
        &= \left[g^{\sigma}(l) \right]^{-1} 
            + \left[ \left(g^{\sigma}(l) \right)^{-1} -1 \right]
            \left(\Delta^{\sigma}_{l} (i) - 1 \right)\\
    \end{aligned}
\]
故:
\[
    \tilde{g}^{\sigma} (l) 
    =   \left[1 + 
            \left( 1 - g^{\sigma} (l) \right) \left(\Delta^{\sigma}_{l} (i) - 1 \right)
        \right] ^{-1}
        g^{\sigma}(l)
\]
我们定义矩阵
\[
    \Gamma_{l}^{\sigma} (l) \equiv \Delta^{\sigma}_{l} (i) - 1
\]
注意这个矩阵只有一个矩阵元$[\Gamma_{l}^{\sigma}(i)]_{ii} = \ee^{2\sigma \lambda s_{i}(l)}-1$
不为零,其余矩阵元都为零。我们定义$\gamma_{i}^{\sigma}(l) \equiv [\Gamma_{l}^{\sigma}(i)]_{ii}$
则:
\[
    \tilde{g}^{\sigma} (l) 
    =   \left[1 + 
            \left( 1 - g^{\sigma} (l) \right) \Gamma_{l}^{\sigma} (l)
        \right] ^{-1}
        g^{\sigma}(l)
\]
这样$g$矩阵的更新就可以通过一次矩阵求逆和一次矩阵乘法实现
(矩阵求逆和矩阵乘法的时间复杂度大致是$O(n^3)$其中$n$为矩阵的维度),但是考虑到$\Gamma$矩阵的稀疏性,事实上
我们是可以用手求出上式中矩阵的逆的,并且通过逆矩阵的稀疏性我们也可以手求出两个矩阵的乘法的。现在定义
\[
    M = 1 + \left( 1 - g^{\sigma} (l) \right) \Gamma_{l}^{\sigma} (i)
\]
由于$\Gamma$矩阵只有一个元素不为零,所以$M$矩阵只有对角元以及第$i$列不为零,根据逆矩阵的基本求法
\[
    [A^{-1}]_{ij} = \frac{(-1)^{j + i}A_{ji}}{|A|}
\]
我们只需要求出$M$矩阵的行列式,以及非零的代数余子式就可以了。$M$的行列式很简单:
$|M| = 1 + (1 - g_{ii}^{\sigma} (l)) \gamma_{l}^{\sigma}(i)$
$M$非零的代数余子式可以分为三种:1.
$(-1)^{j + j} M_{jj} = 1 + (1 - g_{ii}^{\sigma} (l)) \gamma_{l}^{\sigma}(i) = |M|
    \quad \mathrm{for} j \neq i$
2.
$(-1)^{i + j} M_{ij} = -(\delta_{ji} - g_{ji}^{\sigma}(l)) \gamma^{\sigma}_{l}(i)
\quad \mathrm{for} j \neq i$ 3. $(-1)^{i+i}M_{ii} = 1$ 
综上:
\[
    \begin{aligned}
        (-1)^{j + k} M_{jk}=& |M|\delta_{jk} - |M|\delta_{ji}\delta_{ki}\\
            & - \delta_{ji} (\delta_{ki} - g_{ki}^{\sigma}(l)) \gamma_{l}^{\sigma}(i)\\
            & + \delta_{ji} \delta_{ki}(1 - g_{ii}^{\sigma}(l)) \gamma_{l}^{\sigma}(i)\\
            &+\delta_{ji}\delta_{ki}\\
            =& |M|\delta_{jk} - \delta_{ji} (\delta_{ki} - g_{ki}^{\sigma}(l)) \gamma_{l}^{\sigma}(i)
    \end{aligned}
\]
所以
\[
    [M^{-1}]_{jk} = \delta_{jk} - \frac{\delta_{ki}(\delta_{ji} 
    - g_{ji}^{\sigma}(l))\gamma_{l}^{\sigma}(i)}{|M|}   
\]
所以
\[
    [\tilde{g}^{\sigma} (l)]_{jk} = g_{jk}^{\sigma}(i)
    - \frac{(\delta_{ji} 
    - g_{ji}^{\sigma}(l))\gamma_{l}^{\sigma}(i) g_{ik}^{\sigma}(l)}{|M|}
\]
\appendix
\section{需要用到的一些命题的证明}
\label{bookproof}
首先我们给出一些符号约定。对于一个有$N_s$个单粒子态的系统我们把产生湮灭算符写成矢量的形式:
\[
    \mathbf{c}=
        \left(
            \begin{aligned}
                &c_{1}\\
                &c_{2}\\
                &\vdots\\
                &c_{N_s}
            \end{aligned}
        \right),
    \qquad
    \mathbf{c^{\dag}}=
        \left(
            c^{\dag}_{1} \quad c^{\dag}_{2} \quad \cdots \quad c^{\dag}_{N_s}
        \right)
\]
下文中出现的$P$矩阵,是$N_s\times N_p$维的矩阵。%且$\sum_{i}P^{*}_{in}P_{im}=\delta_{nm}$。

\noindent 命题一:
\[
    \ee^{\mathbf{c^{\dag}}T \mathbf{c}} \prod_{n=1}^{N_p}\left( \mathbf{c^{\dag}} P\right)_{n}
        \left|0\right>
    =\prod_{n=1}^{N_p}\left( \mathbf{c^{\dag}} \ee^{T} P\right)_{n}\left|0\right>
\]
其中 $T$ 是厄米矩阵。

\noindent 证:厄米矩阵总是可以写成如下的形式:$T=U \Lambda U^{\dag}$,其中$\Lambda$是对角矩阵,$U$是酉矩阵。
我们假设$\boldsymbol{\gamma}^{\dag}=\mathbf{c^{\dag}}U$,我们可以很容易验证$\gamma_n$满足费米子算符的
反对易关系,$\{\gamma_m,\gamma_n\}=\{\gamma^{\dag}_m,\gamma^{\dag}_n\}=0$,$\{\gamma_m,\gamma^{\dag}_n\}=\delta_{mn}$。
\[
    \begin{aligned}
        &\ee^{\mathbf{c^{\dag}}T \mathbf{c}} \prod_{n=1}^{N_p}\left( \mathbf{c^{\dag}} P\right)_{n}
            \left|0\right>
        =\ee^{\boldsymbol{\gamma^{\dag}} \Lambda \boldsymbol{\gamma}} \prod_{n=1}^{N_p}
            \left(\boldsymbol{\gamma^{\dag}} U^{\dag} P\right)_n  \left|0\right>\\
        =&\sum_{y_1,y_2, \cdots, y_{N_p} }\ee^{\sum_{i}\gamma^{\dag}_i \lambda_i \gamma_i}  \gamma^{\dag}_{y_1} \gamma^{\dag}_{y_2}
        \cdots \gamma^{\dag}_{y_{N_p}} \left|0\right> \left(U^{\dag} P\right)_{y_1 1}
        \left(U^{\dag} P\right)_{y_2 2} \cdots \left(U^{\dag} P\right)_{y_{N_p} N_p}\\
        =&\sum_{y_1,y_2, \cdots, y_{N_p} } \gamma^{\dag}_{y_1} \ee^{\lambda_{y_1}} \gamma^{\dag}_{y_2}
        \ee^{\lambda_{y_2}}\cdots \gamma^{\dag}_{y_{N_p}} \ee^{\lambda_{y_{N_p}}} \left|0\right> \left(U^{\dag} P\right)_{y_1 1}
        \left(U^{\dag} P\right)_{y_2 2} \cdots \left(U^{\dag} P\right)_{y_{N_p} N_p}\\
        =&\prod_{n=1}^{N_p} \left(\boldsymbol{\gamma^{\dag}}\ee^{\Lambda} U^{\dag} P\right)_n \left|0\right>
        =\prod_{n=1}^{N_p} \left(\mathbf{c^{\dag}} U \ee^{\Lambda} U^{\dag} P\right)_n \left|0\right>
        =\prod_{n=1}^{N_p}\left( \mathbf{c^{\dag}} \ee^{T} P\right)_{n}\left|0\right>
    \end{aligned}
\]
命题二:令
\[
    \begin{aligned}
        \left| \Psi \right> &= \prod_{n=1}^{N_p}\left( \mathbf{c^{\dag}} P\right)_{n} \left|0\right> \\
        \left| \tilde{\Psi} \right> &= \prod_{n=1}^{N_p}\left( \mathbf{c^{\dag}} \tilde{P}\right)_{n} \left|0\right>
    \end{aligned} 
\]
则
\[
    \left< \Psi \left| \tilde{\Psi} \right. \right> = \det\left( P^{\dag} \tilde{P}\right)    
\]
证:
\[
    \begin{aligned}
        \left< \Psi \left| \tilde{\Psi} \right. \right>
        &=\left< 0 \right| \prod_{n=N_p}^{1} \left( P^{\dag} \mathbf{c} \right)_n
            \prod_{n'=1}^{N_p} \left( \mathbf{c^{\dag}} \tilde{P}\right)_{n} \left|0\right>\\
        &=\sum_{\mbox{\tiny$\begin{array}{c}
            x_1,x_2,\cdots,x_{N_p}\\
            y_1,y_2,\cdots,y_{N_p}
           \end{array}$}}
        \left< 0 \right| c_{x_{N_p}} \cdots c_{x_1} c^{\dag}_{y_1} \cdots c^{\dag}_{y_{N_p}} \left| 0 \right>\\
        &\times  \left(P^{\dag}\right)_{N_p x_{N_p}} \cdots \left(P^{\dag}\right)_{1 x_{1}}
        \tilde{P}_{y_1 1} \cdots \tilde{P}_{y_{N_p} N_p}\\
        &=\sum_{x_1,x_2,\cdots,x_{N_p},\pi}(-1)^{\pi} 
        \left(P^{\dag}\right)_{N_p x_{N_p}} \cdots \left(P^{\dag}\right)_{1 x_{1}}
        \tilde{P}_{x_{\pi(1)} 1} \cdots \tilde{P}_{x_{\pi(N_p)} N_p}\\
        &=\sum_{x_1,x_2,\cdots,x_{N_p},\pi}(-1)^{\pi^{-1}} 
        \left(P^{\dag}\right)_{N_p x_{N_p}} \cdots \left(P^{\dag}\right)_{1 x_{1}}
        \tilde{P}_{x_1 \pi^{-1}(1)} \cdots \tilde{P}_{x_{N_p} \pi^{-1}(N_p)}\\
        &=\sum_{\pi} (-1)^\pi \left( P^{\dag} \tilde{P}\right)_{1,\pi(1)} 
        \left( P^{\dag} \tilde{P}\right)_{2,\pi(2)}   \cdots
        \left( P^{\dag} \tilde{P}\right)_{N_p,\pi(N_p)} \\
        &=\det(P^{\dag} \tilde{P})
    \end{aligned}    
\]
命题三:
\[
    \mathrm{Tr} \left[ \ee^{\mathbf{c^{\dag}} T_1 \mathbf{c}} 
    \ee^{\mathbf{c^{\dag}} T_2 \mathbf{c}} \cdots \ee^{\mathbf{c^{\dag}} T_n \mathbf{c}} \right]
    = \det \left[ 1+ \ee^{T_1} \ee^{T_2} \cdots \ee^{T_n}\right] 
\]
证:令$U=\ee^{\mathbf{c^{\dag}} T_1 \mathbf{c}} 
\ee^{\mathbf{c^{\dag}} T_2 \mathbf{c}} \cdots \ee^{\mathbf{c^{\dag}} T_n \mathbf{c}}$,
$B=\ee^{T_1} \ee^{T_2} \cdots \ee^{T_n}$
\[
    \begin{aligned}
        &\det(1+B)\\
        =&\sum_{\pi} (-1)^{\pi} \left(B_{1,\pi(1)} + \delta_{1,\pi(1)}\right)
        \left(B_{2,\pi(2)} + \delta_{2,\pi(2)}\right) \cdots
        \left(B_{N_s,\pi(N_s)} + \delta_{N_s,\pi(N_s)}\right)\\
        =&\sum_{\pi} (-1)^{\pi} \delta_{1,\pi(1)} \delta_{2,\pi(2)} \cdots \delta_{N_s,\pi(N_s)}\\
        &+\sum_{x}\sum_{\pi} (-1)^{\pi} B_{x,\pi(x)} \delta_{1,\pi(1)} \delta_{2,\pi(2)}  \cdots  
        \overbrace{\delta_{x,\pi(x)}} \cdots \delta_{N_s,\pi(N_s)}\\
        &+\sum_{x<y} \sum_{\pi} (-1)^{\pi}  B_{x,\pi(x)}  B_{y,\pi(y)} \delta_{1,\pi(1)}   \cdots  
        \overbrace{\delta_{x,\pi(x)}} \cdots \overbrace{\delta_{y,\pi(y)}} \cdots
        \delta_{N_s,\pi(N_s)}\\
        &+\sum_{x<y<z} \sum_{\pi} (-1)^{\pi} B_{x,\pi(x)}  B_{y,\pi(y)} B_{z,\pi(z)} \delta_{1,\pi(1)}   \cdots  
        \overbrace{\delta_{x,\pi(x)}} \cdots \overbrace{\delta_{y,\pi(y)}} \cdots
        \overbrace{\delta_{z,\pi(z)}} \cdots \delta_{N_s,\pi(N_s)}\\
        &+\cdots
    \end{aligned}    
\]
其中$\overbrace{\delta_{x,\pi(x)}}$代表这个连乘中没有这一项。
考察前几个项:
\[
    \sum_{\pi} (-1)^{\pi} \delta_{1,\pi(1)} \delta_{2,\pi(2)} \cdots \delta_{N_s,\pi(N_s)}=1 
\]
\[
    \begin{aligned}
        &\sum_{\pi} (-1)^{\pi} B_{x,\pi(x)} \delta_{1,\pi(1)} \delta_{2,\pi(2)}  \cdots  
        \overbrace{\delta_{x,\pi(x)}} \cdots \delta_{N_s,\pi(N_s)}\\
        =&B_{x,x}\\
        =&\det\left(P(x)^{\dag}BP(x)\right)\\
        =&\left< 0 \right| \left(P(x)^{\dag} \mathbf{c}\right) \left(\mathbf{c^{\dag}}B P(x)\right)
        \left| 0 \right>\\
        =&\left< 0 \right| \left(P(x)^{\dag} \mathbf{c}\right) U \left(\mathbf{c^{\dag}} P(x))\right)
        \left| 0 \right>\\
        =&\left< 0 \right| c_{x} U c^{\dag}_{x} \left| 0 \right>
    \end{aligned}
\]
\[
    \begin{aligned}
        &\sum_{\pi} (-1)^{\pi}  B_{x,\pi(x)}  B_{y,\pi(y)} \delta_{1,\pi(1)}   \cdots  
        \overbrace{\delta_{x,\pi(x)}} \cdots \overbrace{\delta_{y,\pi(y)}} \cdots
        \delta_{N_s,\pi(N_s)}\\
        =&B_{x,x}B_{y,y}-B{x,y}B{y,x}\\
        =&\det\left(P(x,y)^{\dag}BP(x,y)\right)\\
        =&\left< 0 \right| \prod_{n=2}^{1}\left(P(x,y)^{\dag} \mathbf{c}\right)_n 
        \prod_{n'=1}^{2}\left(\mathbf{c^{\dag}}B P(x,y)\right)_{n'} \left| 0 \right>\\
        =&\left< 0 \right| \prod_{n=2}^{1} \left(P(x,y)^{\dag} \mathbf{c}\right)_{n} U 
        \prod_{n'=1}^{2}\left(\mathbf{c^{\dag}} P(x,y))\right)_{n'} \left| 0 \right>\\
        =&\left< 0 \right| c_{y} c_{x} U c^{\dag}_{x} c^{\dag}_{y} \left| 0 \right>
    \end{aligned}    
\]
上面推导利用了定理一和定理二。其中$P(x)$代表$N_s\times 1$的矩阵,其中只有第$x$行的矩阵元为1,其余矩阵元为零。
$P(x,y)$代表$N_s\times 2$的矩阵,第一列中只有第$x$行的矩阵元为1,其余矩阵元为零,第二列中只有第$y$行的矩阵元为1,其余矩阵元为零。
重复上述的讨论我们可以得到:
\[
    \begin{aligned}
        &\det(1+B)\\
        =&1+\sum_{x} \left< 0 \right| c_{x} U c^{\dag}_{x} \left| 0 \right>
        +\sum_{x<y} \left< 0 \right| c_{y} c_{x} U c^{\dag}_{x} c^{\dag}_{y} \left| 0 \right>\\
        &+\sum_{x<y<z} \left< 0 \right| c_{z} c_{y} c_{x} U c^{\dag}_{x} c^{\dag}_{y} c^{\dag}_{z} \left| 0 \right>
        +\cdots\\
        =&\mathrm{Tr}[U]
    \end{aligned}    
\]
即
\[
    \mathrm{Tr} \left[ \ee^{\mathbf{c^{\dag}} T_1 \mathbf{c}} 
    \ee^{\mathbf{c^{\dag}} T_2 \mathbf{c}} \cdots \ee^{\mathbf{c^{\dag}} T_n \mathbf{c}} \right]
    = \det \left[ 1+ \ee^{T_1} \ee^{T_2} \cdots \ee^{T_n}\right] 
\]
命题三得证。
\iffalse
\section{附录\ref{bookproof}中命题三的另一种证明}
首先需要指出需要用到的一个定理(见Lie Groups, Lie Algebras, and Representations定理2.10):
\begin{quote}
    任意可逆的$n \times n$矩阵都可以表示为一些$n \times n$矩阵$X$的指数$\ee^{X}$。
\end{quote}
我们先证明一个命题:
\[
    \left< i \right| \ee^{\mathbf{c^{\dag}} T \mathbf{c}} \left| j \right>
    =\left[ \ee^{T} \right]_{ij}
\]
证:将$T$对角化即$T=U \Lambda U^{\dag}$ 其中$U$是酉矩阵,$\Lambda$是对角矩阵且$\Lambda_{ii}=\lambda_{i}$,我们令
$\boldsymbol{\gamma^{\dag}}=\mathbf{c^{\dag}} U$,$\boldsymbol{\gamma}=U^{\dag} \mathbf{c}$。
假设$\left| i \right> = c^{\dag}_{i} \left| 0 \right>$,
$\left | \alpha \right> = \gamma^{\dag}_{\alpha} \left| 0 \right>$,则
\[
    \left < i \right| \left. \alpha \right> =\left < i \right|  \gamma^{\dag}_{\alpha} \left| 0 \right>
    = \sum_{j} \left < i \right| c^{\dag}_{j} U_{j \alpha}  \left| 0 \right>
    = U_{i\alpha}
\]
进而
\[
    \begin{aligned}
        &\left< i \right| \ee^{\mathbf{c^{\dag}} T \mathbf{c}} \left| j \right>
        =\left< i \right| \ee^{\sum_{\alpha} \lambda_{\alpha} \gamma^{\dag}_{\alpha} \gamma_{\alpha}} \left| j \right>\\
        =&\sum_{\beta_1,\beta_2} \left< i \left| \beta_1 \right. \right>
        \left< \beta_1 \right| \ee^{\sum_{\alpha} \lambda_{\alpha} \gamma^{\dag}_{\alpha} \gamma_{\alpha}} 
        \left| \beta_2 \right>  \left<\beta_2 \left| j \right. \right> \\
        =&\sum_{\beta_1,\beta_2} U_{i\beta_1} \ee^{\lambda_{\beta_2}} \delta_{\beta_1 \beta_2}
        U^{*}_{j\beta_2}\\
        =&\left[ U \ee^{\Lambda} U^{\dag} \right]_{ij} = \left[ \ee^{T} \right]_{ij}
    \end{aligned}    
\]
\fi
\end{document}